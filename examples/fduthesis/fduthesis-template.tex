\documentclass[type=doctor, oneside]{fduthesis}

% 文本来源:
% https://zh.wikipedia.org/wiki/量子力学
% https://zh.wikipedia.org/wiki/量子力学入门
% https://en.wikipedia.org/wiki/Introduction_to_quantum_mechanics

\fdusetup{
  style = {
    font-size = 5,
    bib-backend = bibtex,
    bib-resource = {fduthesis-template.bib},
    auto-make-cover = false,
  },
  info = {
    title = {论文标题},
    title* = {Thesis Title},
    author = {王二},
    supervisor = {某某某\quad 教授},
    major = {物理学},
    degree = academic,
    department = {物理系},
    student-id = {18110190000},
    keywords = {不确定关系, 量子力学, 理论物理},
    keywords* = {Uncertainty principle, quantum mechanics, theoretical physics},
    clc = {O413.1},
  }
}

\usepackage{physics}

\newcommand{\hilbertH}{\symcal{H}}
\newcommand{\ee}{\symrm{e}}
\newcommand{\ii}{\symrm{i}}

\begin{document}

\begin{titlepage}
  \makecoveri
\end{titlepage}

\frontmatter

\tableofcontents

\begin{abstract}
量子力学是描述微观物质(原子、亚原子粒子)行为的物理学理论,量子力学是我们理解除万有引力之外的
所有基本力(电磁相互作用、强相互作用、弱相互作用)的基础。

量子力学是许多物理学分支的基础,包括电磁学、粒子物理、凝聚态物理以及宇宙学的部分内容。量子力学
也是化学键理论、结构生物学以及电子学等学科的基础。

量子力学主要是用来描述微观下的行为,所描述的粒子现象无法精确地以经典力学诠释。例如:根据哥本哈根
诠释,一个粒子在被观测之前,不具有任何物理性质,然而被观测之后,依测量仪器而定,可能观测到其粒子
性质,也可能观测到其波动性质,或者观测到一部分粒子性质一部分波动性质,此即波粒二象性。

量子力学始于 20 世纪初马克斯·普朗克和尼尔斯·玻尔的开创性工作,马克斯·玻恩于 1924 年创造了
“量子力学”一词。因其成功的解释了经典力学无法解释的实验现象,并精确地预言了此后的一些发现,物理学界
开始广泛接受这个新理论。量子力学早期的一个主要成就是成功地解释了波粒二象性,此术语源于亚原子粒子
同时表现出粒子和波的特性。
\end{abstract}

\begin{abstract*}
Quantum mechanics is a fundamental theory in physics that provides a description of the physical
properties of nature at the scale of atoms and subatomic particles. It is the foundation of all
quantum physics including quantum chemistry, quantum field theory, quantum technology, and quantum
information science.

Classical physics, the collection of theories that existed before the advent of quantum mechanics,
describes many aspects of nature at an ordinary (macroscopic) scale, but is not sufficient for
describing them at small (atomic and subatomic) scales. Most theories in classical physics can be
derived from quantum mechanics as an approximation valid at large (macroscopic) scale.

Quantum mechanics differs from classical physics in that energy, momentum, angular momentum, and
other quantities of a bound system are restricted to discrete values (quantization), objects have
characteristics of both particles and waves (wave--particle duality), and there are limits to how
accurately the value of a physical quantity can be predicted prior to its measurement, given a
complete set of initial conditions (the uncertainty principle).

Quantum mechanics arose gradually from theories to explain observations which could not be
reconciled with classical physics, such as Max Planck's solution in 1900 to the black-body
radiation problem, and the correspondence between energy and frequency in Albert Einstein's 1905
paper which explained the photoelectric effect. These early attempts to understand microscopic
phenomena, now known as the ``old quantum theory'', led to the full development of quantum
mechanics in the mid-1920s by Niels Bohr, Erwin Schrödinger, Werner Heisenberg, Max Born and
others. The modern theory is formulated in various specially developed mathematical formalisms.
In one of them, a mathematical entity called the wave function provides information, in the form
of probability amplitudes, about what measurements of a particle's energy, momentum, and other
physical properties may yield.
\end{abstract*}

\mainmatter

\chapter{数学基础}

\strong{量子力学}是物理学的分支学科。它主要描写微观的事物,与相对论一起被认为是现代物理学的两大
基本支柱,许多物理学理论和科学,如原子物理学、固体物理学、核物理学和粒子物理学以及其它相关的学科,
都是以其为基础\cite{曾谨言2013量子力学,feynman2011feynman}。

\section{基础公设}

整个量子力学的数学理论可以建立于五个基础公设。这些公设不能被严格推导出来的,而是从实验结果仔细分析
归纳总结而得到的。从这五个公设,可以推导出整个量子力学。假若量子力学的理论结果不符合实验结果,
则必须将这些基础公设加以修改,直到没有任何不符合之处。至今为止,量子力学已被实验核对至极高准确度,
还没有找到任何与理论不符合的实验结果,虽然有些理论很难直觉地用经典物理的概念来理解,例如,波粒
二象性、量子纠缠等等\cite{zurek2014quantum,cohen2013claude,zettili2003quantum}。

\begin{enumerate}
  \item 量子态公设:量子系统在任意时刻的状态(量子态)可以由希尔伯特空间 $\hilbertH$ 中的态矢量
    $\ket{\psi}$ 来设定,这态矢量完备地给出了这量子系统的所有信息。这公设意味着量子系统遵守%
    \emph{态叠加原理},假若 $\ket*{\psi_1}$、$\ket*{\psi_2}$ 属于希尔伯特空间 $\hilbertH$,则
    $c_1\ket*{\psi_1} + c_2\ket*{\psi_2}$ 也属于希尔伯特空间 $\hilbertH$。
  \item 时间演化公设: 态矢量为 $\ket{\psi(t)}$ 的量子系统,其动力学演化可以用薛定谔方程表示:
    \begin{equation}
      \ii\hbar \pdv{t} \ket{\psi(t)} = \hat{H} \ket{\psi(t)}.
    \end{equation}
    其中,哈密顿算符 $\hat{H}$ 对应于量子系统的总能量,$\hbar$ 是约化普朗克常数。根据薛定谔方程,
    假设时间从 $t_0$ 变化到 $t$,则态矢量从 $\ket*{\psi(t_0)}$ 演化到 $\ket{\psi(t)}$,该过程以
    方程表示为
    \begin{equation}
      \ket{\psi(t)} = \hat{U}(t,\,t_0) \ket*{\psi(t_0)}.
    \end{equation}
    其中 $\hat{U}(t,\,t_0) = \ee^{-\ii\hat{H}(t-t_0) / \hbar}$ 是时间演化算符。
  \item 可观察量公设:每个可观察量 $A$ 都有其对应的厄米算符 $\hat{A}$,而算符 $\hat{A}$ 的所有
    本征矢量共同组成一个完备基底。
  \item 坍缩公设:对于量子系统测量某个可观察量 $A$ 的过程,可以数学表示为将对应的厄米算符
    $\hat{A}$ 作用于量子系统的态矢量 $\ket{\psi}$,测量值只能为厄米算符 $\hat{A}$ 的本征值。
    在测量后,假设测量值为 $a_i$,则量子系统的量子态立刻会坍缩为对应于本征值 $a_i$ 的本征态
    $\ket*{e_i}$。
  \item 波恩公设:对于这测量,获得本征值 $a_i$ 的概率为量子态 $\ket{\psi}$ 处于本征态 $\ket*{e_i}$
    的概率幅的绝对值平方。\footnote{%
      使用可观察量 $A$ 的基底 $\qty{e_1,\,e_2,\,\ldots,\,e_n}$,量子态 $\ket{\psi}$ 可以表示为
      $\ket{\psi} = \sum_j c_j \ket*{e_j}$,其中 $c_j$ 是量子态 $\ket{\psi}$ 处于本征态
      $\ket*{e_j}$ 的概率幅。根据波恩定则,对于此次测量,获得本征值 $a_i$ 的概率为
      $\abs*{\ip*{e_i}{\psi}}^2 = \abs*{c_i}^2$。}
\end{enumerate}

\section{量子态与量子算符}

量子态指的是量子系统的状态,态矢量可以用来抽象地表现量子态。采用狄拉克标记,态矢量表示为右矢
$\ket{\psi}$;其中,在符号内部的希腊字母 $\psi$ 可以是任何符号、字母、数字,或单字。例如,
沿着磁场方向测量电子的自旋,得到的结果可以是上旋或是下旋,分别标记为 $\ket{\uparrow}$ 和
$\ket{\downarrow}$。

\begin{figure}[htb]
  \centering
  \setlength{\fboxsep}{4em}
  \fbox{\texttt{stern-gerlach-experiment.png}}
  \caption[施特恩—格拉赫实验]{%
    设定施特恩—格拉赫实验仪器的磁场方向为 $z$-轴,入射的银原子束可以被分裂成两道银原子束,每一道
    银原子束代表一种量子态,上旋 $\ket{\uparrow}$ 或下旋 $\ket{\downarrow}$%
    \cite{wikimedia:stern-gerlach-experiment}。}
  \label{fig:stern-gerlach-experiment}
\end{figure}

对量子态做操作定义,量子态可以从一系列制备程序来辨认,即这程序所制成的量子系统拥有这量子态。例如,
使用施特恩—格拉赫实验仪器,设定磁场朝着 $z$-轴方向,如图~\ref{fig:stern-gerlach-experiment} 所示,
可以将入射的银原子束,依照自旋的 $z$-分量分裂成两道,一道为上旋,量子态为 $\ket{\uparrow}$;另一道
为下旋,量子态为 $\ket{\downarrow}$,这样,可以制备成量子态为 $\ket{\uparrow}$ 的银原子束,或量子态
为 $\ket{\downarrow}$ 的银原子束。原本银原子束的态矢量可以按照态叠加原理表示为
\begin{equation}
  \ket{\psi} = \alpha \ket{\uparrow} + \beta \ket{\downarrow}.
\end{equation}
其中,$\alpha$、$\beta$ 是复值系数,$\abs{\alpha}^2$、$\abs{\beta}^2$ 分别为入射银原子束处于上旋、
下旋的概率,且有
\begin{equation}
  \abs{\alpha}^2 + \abs{\beta}^2 = 1.
\end{equation}

\section{动力学演化}

量子系统的动力学演化可以用不同的绘景来表现。通过重新定义,这些不同的绘景可以互相变换,它们实际上
是等价的。假若要专注分析量子态怎样随着时间的流易而演化,时间演化算符怎样影响量子态,则可采用
薛定谔绘景。假若要专注了解对应于可观察量的算符怎样随着时间的流易而演化、时间演化算符怎样影响这些
算符,则可采用海森堡绘景。

\backmatter

\printbibliography

\makecoveriii

\end{document}
