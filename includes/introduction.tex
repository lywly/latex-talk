\begin{frame}[standout]
  \large\XKCD
  ``Your paper makes no goddamn sense, \\
  but it's the most beautiful thing \\
  \XeTeXglyph\XeTeXglyphindex"I_hyphen_p_r_o_n_o_u_n"\relax\
  have ever laid eyes on.''

  \small
  \hfill From r/ProgrammerHumor
  \href{https://www.reddit.com/r/ProgrammerHumor/comments/2jf7yl}{\faRedditAlien}
\end{frame}

\section{介绍}

\begin{frame}{历史回眸}
\begin{columns}
\begin{column}{0.45\textwidth}
  \begin{figure}
    \centering
    \includegraphics[height=3.2cm]{images/knuth-2018.jpg}
    \caption{高德纳(Donald~E. Knuth) \\ \TeX}
  \end{figure}
\end{column}
\begin{column}{0.45\textwidth}
  \begin{figure}
    \centering
    \includegraphics[height=3.2cm]{images/lamport-2018.jpg}
    \caption{Leslie Lamport \\ \LaTeX}
  \end{figure}
\end{column}
\end{columns}
\nonumberfootnote{图片来源:
  \link{https://www-cs-faculty.stanford.edu/~knuth/graphics.html}
  \link{https://aperiodical.com/2018/09/hlf-blogs-leslie-lamport-thinks-your-code-is-bad}}
\end{frame}

\begin{frame}{\LaTeX{} 是什么?}
\pause
\begin{itemize}
  \item<+-> 发音:

    \begin{itemize}
      \item /ˈlɑːtɛx/ or /ˈleɪtɛx/ or whatever you like
    \end{itemize}

  \item<+-> 打公式方便?

    \begin{itemize}
      \item 「复杂公式输入哪家强,当然首选 \LaTeX{} 帮忙」
    \end{itemize}

  \item<+-> 写论文神器?

    \begin{itemize}
      \item 「想要轻松给论文排版,当然少不了 \LaTeX{} 啦」
    \end{itemize}

  \item<+-> 不想做宏编程的标记语言不是好的排版引擎?

    \begin{itemize}
      \item {\tiny
        \LaTeX{} is a high-quality typesetting system; it includes features
        designed for the production of technical and scientific documentation.
        \LaTeX{} is the \textit{de facto} standard for the communication and
        publication of scientific documents. \LaTeX{} is available as free
        software. \link{https://www.latex-project.org}}
    \end{itemize}
\end{itemize}
\end{frame}

\begin{frame}[fragile]
\frametitle{\LaTeX{} 是什么?\mbox{}——\mbox{}What you \emph{think} is what you get!}
\begin{columns}
\begin{column}{0.5\textwidth}
  \begin{texcode}[basicstyle=\tiny\ttfamily, moretexcs={\maketitle},
    emph={[1]equation,itemize,document}, emph={[2]article,amsmath,graphicx}]
  \documentclass{article}
  \usepackage{amsmath,graphicx}
  \title{Normal distribution}
  \author{Wikipedia, the free encyclopedia}

  \begin{document}
  \maketitle
  \section{Introduction}
  % 省略一些内容……
  The probability density of the normal
  distribution is
  \begin{equation}
    f(x|\mu, \sigma)
    = \frac{1}{\sqrt{2\pi\sigma^2}}
      e^{-\frac{(x-\mu)^2}{2\sigma^2}}
  \end{equation}
  where
  \begin{itemize}
    \item $\mu$ is the mean of the distribution
    \item $\sigma$ is the standard deviation
  \end{itemize}
  \end{document}
  \end{texcode}
\end{column}
\pause
\begin{column}{0.42\textwidth}
  \begin{figure}
    \centering
    \vspace{-0.8cm}
    \includegraphics[width=\textwidth, trim={2cm 2cm 2cm 2cm}, clip]%
      {examples/normal-dist/normal-dist.pdf}
  \end{figure}
\end{column}
\end{columns}
\nonumberfootnote{来源:Wikipedia \link{https://en.wikipedia.org/wiki/Normal_distribution}}
\end{frame}

\begin{frame}{基本原则}
\begin{itemize}
  \item<+-> 排版 vs 文字处理

    \begin{itemize}
      \item 《别把 \LaTeX{} 当 Word 用》
    \end{itemize}

  \item<+-> 遵循业\zhparen{xué}界\zhparen{xiào}规范

    \begin{itemize}
      \item 《管教务处 or 研究生院 or 物理系叫爸爸》
    \end{itemize}

  \item<+-> 追求良好的阅读体验\zhparen{readability}
  \item<+-> 内容与格式分离
  \item<+-> \alert{内容永远比格式重要!}

    \begin{itemize}
      \item \emph{Typography exists to honor content.} ---R. Bringhurst
    \end{itemize}
\end{itemize}
\end{frame}
